\documentclass[10pt,conference]{IEEEtran}
\IEEEoverridecommandlockouts
% The preceding line is only needed to identify funding in the first footnote. If that is unneeded, please comment it out.
\usepackage{cite}
\usepackage{comment}
\usepackage{amsmath,amssymb,amsfonts}
\usepackage{algorithmic}
\usepackage{graphicx}
\usepackage{textcomp}
\usepackage{xcolor}
\newcommand{\timedEffects}{\emph{TimEffs}}
\newcommand\figref[1]{Fig. \textcolor{black}{\ref{#1}}.}
\newcommand\tabref[1]{Table \textcolor{black}{\ref{#1}}}
\newcommand\secref[1]{Sec. \textcolor{black}{\ref{#1}}}
\newcommand{\timedL}{\code{C^{t}}}
\newcommand{\anyevent}[1]{{\textcolor{darkred}
{{\textbf{\small #1}}}}}
\newcommand{\anynotevent}[1]{{\textcolor{darkred}
{{\textbf{\footnotesize $\overline{\text{#1}}$}}}}}
\newcommand{\code}[1]{{\tt{\ensuremath{\m{#1}}}}}
\newcommand{\codeme}[1]{{\tt{\ensuremath{#1}}}}
\newcommand{\CONTAIN}{\sqsubseteq}
\newcommand{\underscore}{\textcolor{vividauburn}{\code{\_}}}
\newcommand{\m}{\mathit} 
\newcommand{\lappend}{\mathrel{\texttt{++}}}

\newcommand{\mysharp}{{\mathrel{\texttt{\#}}}}


\newcommand\theoref[1]{Theorem~\textcolor{blue}{\ref{#1}}}
\newcommand\lemmaref[1]{Lemma~\textcolor{blue}{\ref{#1}}}
\newcommand\appref[1]{Appendix~\textcolor{blue}{\ref{#1}}}
\newcommand\defref[1]{Definition~\textcolor{blue}{\ref{#1}}}
\newcommand\algoref[1]{Algorithm~\textcolor{blue}{\ref{#1}}}


\def\BibTeX{{\rm B\kern-.05em{\sc i\kern-.025em b}\kern-.08em
    T\kern-.1667em\lower.7ex\hbox{E}\kern-.125emX}}
\begin{document}

\title{Automated Verification for Real-Time Systems 
using Implicit Clocks and an Extended Antimirov Algorithm}

\author{(Anonymous Authors)}

\begin{comment}
\author{\IEEEauthorblockN{1\textsuperscript{st} Given Name Surname}
\IEEEauthorblockA{\textit{dept. name of organization (of Aff.)} \\
\textit{name of organization (of Aff.)}\\
City, Country \\
email address or ORCID}
\and
\IEEEauthorblockN{2\textsuperscript{nd} Given Name Surname}
\IEEEauthorblockA{\textit{dept. name of organization (of Aff.)} \\
\textit{name of organization (of Aff.)}\\
City, Country \\
email address or ORCID}
}
\end{comment}

\maketitle

\begin{abstract}
    The correctness of real-time systems 
    depends both on the correct functionalities and the realtime constraints.
    To go beyond the existing Timed Automata based techniques, 
    we propose a novel solution that integrates a 
    modular Hoare-style forward verifier with a new term rewriting 
    system (TRS) on \emph{Timed Effects} (\timedEffects).
    The main purposes are to increase the expressiveness,  
    dynamically create clocks, 
    and efficiently solve constraints on the clocks.  
    We formally define 
    a core language \timedL, generalizing the real-time systems, modeled 
    using mutable variables and timed behavioral patterns, 
    such as \emph{delay}, \emph{timeout}, \emph{interrupt}, \emph{deadline}. 
    Secondly, to capture real-time specifications, 
    we introduce \timedEffects, a new effects logic, 
    that extends 
    \emph{Regular Expressions} with dependent
    values and arithmetic constraints.
    Thirdly,  the forward verifier infers temporal behaviors of given 
    \timedL\ programs, expressed in \timedEffects. 
    Lastly, we present a purely algebraic TRS, i.e., an extended \emph{Antimirov algorithm}, to 
    efficiently prove language inclusions between 
     \timedEffects. 
    To demonstrate the feasibility of our proposals, 
    we prototype the verification system; prove its 
    soundness; report on case studies and the experimental results.     
\end{abstract}

\begin{IEEEkeywords}
component, formatting, style, styling, insert
\end{IEEEkeywords}

\section{Introduction}
This document is a model and instructions for \LaTeX.
Please observe the conference page limits. 



\bibliographystyle{ACM-Reference-Format}
\bibliography{references}
%% Citation style
%% Note: author/year citations are required for papers published as an
%% issue of PACMPL.


%%
%% If your work has an appendix, this is the place to put it.
\appendix

\end{document}
